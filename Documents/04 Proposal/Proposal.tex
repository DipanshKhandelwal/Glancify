\documentclass[12pt]{article}
\usepackage{hyperref}
\usepackage{graphicx}
\usepackage{array}
\usepackage{tabu}
\usepackage[table]{xcolor}
 

\setlength{\arrayrulewidth}{1mm}
\setlength{\tabcolsep}{18pt}
\renewcommand{\arraystretch}{1.6}
\renewcommand{\today}{October 05, 2017}


\begin{document}

\begin{titlepage}
	\begin{center}
		        
		%\vspace*{1cm}
		
		\LARGE{\textbf{Proposal}}
		        
		\vspace{1.5cm}
		      
		\LARGE{\textbf{Glancify}} \\
		\small{Version 1.0}
		
					
					
		\vspace{1.5cm}
						
		\large{September 24, 2017}
		
		
		\vspace{2cm}
		         
		
		\textbf{Indian Institute of Information Technology, Vadodara}
						
	\end{center}
\end{titlepage}
\newpage
\textbf{Team members :} \\
\begin{center}
	\begin{tabular}{ |m{10em} m{8em} m{9em}|}
		\hline
		TEAM MEMBER        &   & ID        \\
		\hline
		Himanshu Singhal             &   & 201551014 \\
		Piyush Sikarawal          &   & 201551020 \\
		Saurabh Srivastava              &   & 201551032 \\
	    Deeoak Sandrana     &   & 201551033 \\
		Sakshee Jain    &   & 201551074 \\
		Neelansh Sahai    &   & 201551086 \\ 
		\hline
	\end{tabular}
	
\end{center}

\newpage

\section{Introduction}
\subsection{Problem Statement}
Human life is very hectic these days.People are busy with their works and don't get sufficient time even to have food properly.In these hectic schedules also people show 
                         intrest to surf through different social networking sites.They do spend lots of time on these sites.
                         
\subsection{What is Glancify?}
                       Motivated by above mentioned time constraint,we planned to create a google chrome
                         extension that will arrange all the notifications from different social networking sites
                         in a sorted manner which would help the users to have a quick glance over their notifications
                         which will inturn save their time.
\subsection{Why Glancify?}
To provide updates to the user about all the stuff happening on their social networks 
                         without wasting users time.
                       To make all these updates easily reachable by providing them the compact form of their 
                       notifications on a new tab page.(reachable because the new tab can be opened in any web browser
                       easily)
\section{Scope}
This is the very wide scope project, mainly because there are very few
constraints. Any user who has internet access and is socially active, can
use this product. Also, the fact that social networking platforms are used
all around the globe, the people who belong to the above defined set are
from every corner of the world and thus, the scope of this project covers
a wide range of users.

\section{Deliverables}
\begin{itemize}
    \item "Select the Media" popup with the list of various Social Networking Platforms. 
    \item "ADD" button to "Select the Media" Procedure.
    \item "Login" button to add login to Social Media Plateform and add that into a "Card".
    \item "Remove" button to remove "Card" and logs user out of the Social Networking Platform.
    \item Clicking "GOTO" button or "notifications" will redirect user to corresponding social networking platform.
\end{itemize}

\section{Time Frame}
\begin{center}
	{%\rowcolors{}{white!80!yellow!50}{white!70!white!40}
		\begin{tabular}{ | c | c | c |  }
			\hline
			Phase & Work Schedule                         & Duration           \\
			\hline
			First     & Feasibility Report, Project Proposal & 1/09/17 - 24/09/17 \\
			\hline
			Second    & Project Plan          &      25/09/17 - 30/09/17\\ 
			\hline
			
			Second    & SRS        &   01/10/17 - 05/10/17\\ 
			\hline
			
			Second    & System Test Plans \& Design        &   07/10/17 - 17/10/17\\ 
			\hline
			
			Second    & Coding \& Unit Testing       &   17/10/17 - 28/10/17\\ 
			\hline
			
			
			Second    & Testing        &   29/10/17 - 07/11/17\\ 
			\hline
			
			
			Third    & Deployment \& Maintenance &   08/11/17 - 15/11/17\\ 
			\hline

		\end{tabular}
	}	
	\end{center}
\newpage
\section{Team Qualification for Project}

\begin{center}
	%\rowcolors{}{white!80!yellow!50}{white!70!white!40}
		\begin{tabular}{ | c | c |}
			\hline
			Member & Skills       \\
			\hline
			Himanshu Singhal   & Java,C,C++,Python,Java script,Html \\
			
		    Piyush Sikarawal    &  Java,Python,Html CSS,Java Script,Django
                             \\ 
			
		Saurabh    &  Java,Python,Php,HTML,CSS,Json
                            \\ 
			
		Deepak  & Java,Php,Html,CSS,Java script.                           
                           \\ 
			
		Sakshee Jain        &  Java,Python,C++,HTML,CSS
                        \\ 
			
			
		Neelansh Sahai        &  Java,Python,Html and CSS \\ 
			
			\hline
			
		\end{tabular}
		

	\end{center}
	\newpage
	        

\end{document}
